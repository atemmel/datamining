\documentclass[a4paper, titlepage,12pt]{article}
\usepackage[margin=3.7cm]{geometry}
\usepackage[utf8]{inputenc}
\usepackage[T1]{fontenc}
\usepackage[swedish,english]{babel}
\usepackage{csquotes}
\usepackage[hyphens]{url}
\usepackage{amsmath,amssymb,amsthm, amsfonts}
\usepackage[backend=biber,citestyle=ieee]{biblatex}
\usepackage[yyyymmdd]{datetime}

\addbibresource{literature.bib}

\title{Title
\large Course code}
\author{Adam Temmel (adte1700),}

\begin{document}
	\maketitle
	\section{Introduction}\label{sec:introduction}
		\textit{Dota 2} is a team based strategy game that is infamous for its relative complexity compared to competing titles. In this game two teams of five players are pitted against each other in a very peculiar arena with several different minor objectives and one major objective; destroying the base of the enemy team. The first team to destroy the enemy base is crowned as the winner, but this requires that the players have a good understanding of the game and even better teamwork. A defining characteristic of the title is the usage of \textit{heroes}. Each player must choose a hero to play as. This sums up to a total of 10 heroes that end up being played each match. Choosing your hero is a quintessential aspect of playing the game effectively, as certain heroes are generally considered better than others in certain scenarios. Therefore, picking a hero that generally underperforms in most situations is usually a bad choice. Naturally every hero has their place in the game, but some end up being more flexible than others.
	\section{Problem description}\label{sec:problem}
		The authors of this project proposal enjoy playing Dota 2, but (most of them) are moderately bad at the game. Therefore, they believe that by mining data from a large dataset of games played, they perhaps might be able to get a better understanding of what heroes to pick. In particular the following topics are things they wish to study:
		\begin{itemize}
			\item Which hero is the \textit{best}, in terms of total winrate? (Games won/Games played)
				% Måste diskutera attribute i introduction
			\item Which hero from each attribute is the \textit{best}, in terms of total winrate?
			\item Create a model which, upon inputing the chosen heroes of two teams, predicts which team is most likely to win, based upon hero choices alone.
		\end{itemize}
	\section{Preprocessing techniques}
	\section{Datamining techniques}
	\section{Evalutation}
	\printbibliography
\end{document}
